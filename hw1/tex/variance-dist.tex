
\section{Variance of Distance}
Present the variance plots for each \( L_p \) and discuss the results. How the curves are behaving when \( k \) increases? Can you characterize the form of curves? What is the effect of \( p \)? If a curve seems to be converging, tell also the value that it is approaching.



For all the quasinorms, i.e, norms p < 1. The variance grows polynomially. Since the data is a sample of random datapoints, the sample variance is used.

\[
    \text{Var}(X) = \frac{1}{n -1 } \sum_{i=1}^{n} \left(X_i - \mu \right)^2
\]


\begin{equation}
    \text{Var}_{\text{sample}}\left(\|\mathbf{x}\|_p\right) = \lim_{k \to \infty} \left( \frac{1}{N - 1} \sum_{i=1}^{k} \left( \left( \sum_{i=1}^{k} |x_i|^p \right)^{\frac{1}{p}} - \mathbb{E}\left[\left( \sum_{i=1}^{k} |x_i|^p \right)^{\frac{1}{p}}\right] \right)^2 \right)
\end{equation}

Similar to the min-max-mean distance, the variance grows polynomially for all p < 1. When p = 1, the variance grows linearly. And when p => 2, the variance grows proportionally to a \(\frac{1}{\log(k)}\) function. (or polynomial??).

The interesting case was the norms p between 1 and 2. For those norms, the variance goes to infinity but much slower than for the p=1 linear growth.

All variances larger or equal to 2 converge to 0 as k increases.

In summary, the curves with p < 1 can be characterized as polynomial curves. The curves with $ 1 < p < 2 \propto k^{1/p}$

Should we add images??
