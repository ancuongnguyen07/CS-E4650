\section{Conclusion}
Upon examination of the statistical properties, including \emph{min}, \emph{max},
\emph{mean}, \emph{variance}, and \emph{relative contrast}, it can be inferred
that the ``curse of dimensionality'' gives rise to unanticipated behaviors within
high-dimensional spaces. Specifically, the growth in dimensionality can yield
the following consequences:


\begin{itemize}
    \item Convergence of \emph{min}, \emph{mean}, and \emph{max} distances,
    causing the majority of data points to become equidistant.
    \item Decrease in \emph{variance} of distances, making distances of data
    points approach zero.
    \item Approach of \emph{relative contrast} towards zero, thereby
    diminishing the distinction between proximate and distant points.
\end{itemize}

In light of these observations, it becomes evident that the "curse of
dimensionality" poses significant challenges for certain machine learning
algorithms when operating in high-dimensional settings, thus necessitating 
appropriate adjustments to ensure optimal performance.