\section{Methods}
In this experiment, Python was used as the primary programming language, utilizing the
libraries \texttt{numpy} for numerical computations and \texttt{matplotlib} for plotting
and visualization. The randomization interval for generating the data points was the
half-open interval \([0,1[\), excluding the origin. The Lp norms were computed using
a custom function:

\lstinputlisting[language=Python]{code/lp_norm.ipynb}

which supported a range of \(p\) values, including p = 0.5, 0.8, 1, 1.3, 1.5, 1.7, 1.9,
1.95, 2, 5, and \(\infty\). For each experiment, \(q = 100\) data sets, each containing
\(N = 100\) points, were generated across various dimensions \((k = 2, 3, 4, 5, 10, 20,
30, 40, 50, 60, 70, 80, 90, 100, 150, 200, 250, 300)\).
