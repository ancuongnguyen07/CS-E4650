\subsection{Improved Algorithm}

- how the algorithm works

- the improvement is the selection of the cliques with the smallest degree.
- the algorithm only works by pruning using the Theorem:

- the theorem formula.
- the only way to improve the algorithm is to improve the pruning
- the pruning happens more often only, if the condition happens more often, and none of the supersets of S, can be a-cliques
- to make the condition true as often as possible, the upper bound on the RHS of the theorem should be maximized
- to maximize that is to minimize the - (negative part) . this is happening, when the deg(v) is minimized (denoting the degree of neighbours in the total graph)
- thus the algorithm should always iterate through neighbours in the order from the smallest degree deg(v) to the largest degree as last.





- the picture of the improvement in a graph
- .png
- explanation if any non-linearities (and thought that whether the alpha value should make a difference there? )
- discussion how the size of the graph would affect the alpha....
