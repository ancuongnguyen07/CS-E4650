\section{Conclusions}

The clustering analysis effectively identified key features that align well with
biological groupings and flying types of birds. The highest normalized mutual
information (\textbf{NMI}) score achieved was 0.727, using a combination of 11 clusters
with features including aspect ratio (\textbf{AR}), wload, wingspan index (\textbf{WSI}), and
back color, and employing complete linkage. Importantly, \textbf{BMI} was consistently
excluded from the top clustering combinations, suggesting its low informative value.

Analysis of the cluster contents revealed strong biological correlations. For
instance, clusters 0, 1, and 2 aligned exclusively with type C, B, and C birds,
respectively, and separated biological families such as podicipedidae and
gaviidae effectively. Hawks, grouped in cluster 7, displayed a high WSI value,
reinforcing their flying prowess. Similarly, birds in cluster 8 were all
dappled brown, showing distinct feature-based clustering.

The results highlight that features like \textbf{AR}, \textbf{wload}, and \textbf{WSI} are significant
in distinguishing bird groups. Future research could explore inter-cluster
relationships and develop systematic methods to categorize cluster contents
based on biological traits. The dendrogram further confirmed the accuracy
of these groupings, visually clustering birds of the same biological
families together using complete linkage.