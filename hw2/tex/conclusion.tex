\section{Conclusions}

The clustering analysis conducted on the dataset revealed that the optimal
combination of features for clustering birds was \emph{AR}, \emph{wload},
\emph{WSI}, and \emph{back} color. Using a complete linkage method, this
achieved a NMI score of 0.663 with the setting of 11 clusters,
indicating a strong similarity between the clustering results and the
true biological groupings. Importantly, \emph{BMI} was consistently
found to provide minimal informative value, as it was excluded from
all of the top 10 cluster combinations.

Cluster analysis revealed distinct patterns: clusters 5 and 7 comprised
only type C fliers, and all birds in cluster 8 were dappled brown.
Hawks were well-grouped in cluster 4, where all members exhibited high
\emph{WSI} values, highlighting the biological intuition behind the
large wingspans of these expert fliers. The dendrogram also accurately
grouped birds of similar biological groups together, further validating
the clusering results.

Overall, the study demonstartes that key features like \emph{AR}, \emph{wload},
and \emph{WSI} are strong indicators of bird classification, while \emph{BMI}
is not. Future work could explore inter-cluster relationships, such as
the proximity of the duck family to other bird groups.