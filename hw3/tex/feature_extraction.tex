\section{Feature Extraction}

Multi-valued features, \textbf{diet} and \textbf{habitat}, were preserved as
they were. In relation to the biology group, level-3 subgroups were merged
into their parent level-2 subgroup. For instance, \emph{dabbling ducks} was
merged into \emph{Anatinae}. By combining these child groups, trivial
association rules were eliminated without applying rule constraints.

For numerical features (\textbf{length, wspan, weight, eggs}), they were
originally represented as ranges. To standardize the data, the \emph{mean} of
each range was calculated, resulting in a single representative value per
feature. This simplification reduced complexity and facilitated straightforward
comparisons between data points. Additionally, new features were engineered to
capture relationships between different measurements. Specifically:

\begin{itemize}
    \item \textbf{Body Mass Index (BMI):} Computed as:
    \[
    \text{BMI} = \frac{\text{weight}}{\text{length}^2}
    \]
    \textbf{BMI} provides a standardized ratio of weight to length, which is
    useful for characterizing the relative bulk of the subjects.
    
    \item \textbf{Wing Span Index (WSI):} Computed as:
    \[
    \text{WSI} = \frac{\text{wspan}}{\text{length}}
    \]
    \textbf{WSI} represents the proportion between wing span and length,
    giving insight into morphological characteristics.
\end{itemize}

To obtain non-trivial association rules, extreme values were considered. As
illustrated in \autoref{tab:extreme-num-vals}, cutoff values were based on the
20th and 80th percentiles. This approach ensured sufficient data points were
filtered for further association rule mining while maintaining their
distinctiveness.

\begin{table}[h]
    \centering
    \begin{tabular}{|p{3cm}|p{6cm}|p{6cm}|}
        \hline
        Feature & Description & Extraction Method \\
        \hline
        \textbf{robust} & If a bird's body is robust, indicated by high \textbf{BMI} &
        True if \textbf{BMI} is higher than the 80th percentile, otherwise False. \\
        \hline
        \textbf{short-winged} & If the bird has short wings, indicated by low 
        \textbf{WSI} & True if \textbf{WSI} is lower than the 20th percentile, 
        otherwise False. \\
        \hline
        \textbf{too-few-eggs} & If a bird has a low number of eggs & True if
        \textbf{eggs} are lower than the 20th percentile, otherwise False. \\
        \hline
        \textbf{too-many-eggs} & If a bird has a high number of eggs & True if
        \textbf{eggs} are higher than the 80th percentile, otherwise False. \\
        \hline
    \end{tabular}
    \caption{Extreme numerical features.}
    \label{tab:extreme-num-vals}
\end{table}

Migration month features, \textbf{arrive} and \textbf{leave}, were transformed
into two new Boolean features: \textbf{early-arrival} and \textbf{late-leave}.
Initially, \textbf{arrive} and \textbf{leave} were represented as strings, 
e.g., Jan-April, which were converted into numerical ranges (1-4). From these
ranges, \textbf{early-arrival} and \textbf{late-leave} were created, as shown
in \autoref{tab:migration-month-feats}. April and September were selected as 
cutoff points for early arrival and late leave, respectively, since most birds
migrate between these months.

\begin{table}[h]
    \centering
    \begin{tabular}{|p{3cm}|p{6cm}|p{6cm}|}
        \hline
        Feature & Description & Extraction Method \\
        \hline
        \textbf{early-arrival} & If a bird arrives in Finland earlier than usual &
        True if the earliest month in \textbf{arrive} is earlier than April, otherwise 
        False. \\
        \hline
        \textbf{late-leave} & If a bird leaves Finland later than usual & True if
        the latest month in \textbf{leave} is later than September, otherwise False. \\
        \hline
    \end{tabular}
    \caption{Migration month features.}
    \label{tab:migration-month-feats}
\end{table}

For multi-valued categorical features, such as \textbf{ftype}, \textbf{incub},
and \textbf{ccare}, each unique value was transformed into a new Boolean 
feature. For example, \textbf{ftype} had three unique values: \emph{A}, \emph{B}, 
and \emph{C}. Hence, three new Boolean features (\textbf{ftype-A}, \textbf{ftype-B}, 
and \textbf{ftype-C}) were created.

For binary features (e.g., "Yes"/"No"), they were converted into Boolean (True/False)
variables for consistency. Additionally, the binary feature \textbf{sim} was decomposed
into two distinct Boolean variables: \textbf{sim-gender}, representing the "Yes" values,
and \textbf{dif-gender}, representing the "No" values. This transformation was performed
to retain both positive and negative instances of the \textbf{sim} feature, as both were
considered informative for the association rule mining process.

All extracted features are presented in \autoref{tab:all-features}.
