%The paper size, font size and document type are defined in the following
\documentclass[a4paper,12pt]{article}

%Uncomment the following line, if you write in Finnish (special characters)
%\usepackage[utf8]{inputenc}

%\usepackage[finnish]{babel}
\usepackage[english]{babel}

\usepackage{graphicx}
\usepackage[parfill]{parskip}

%useful special symbols:
\usepackage{amssymb}
\usepackage{latexsym}
\usepackage{amsmath}
\usepackage{amsthm}

\usepackage[toc,page]{appendix}

%a useful package if you write url addresses:
\usepackage{url}
\usepackage{hyperref}

%a package for figures:
% \usepackage[dvips]{color}
\usepackage{epsfig}

%a package for rotated figures and tables:
\usepackage{rotating}

\usepackage{biblatex}
\addbibresource{references.bib}

\usepackage{listings} % For including code
\usepackage{xcolor}   % For defining code colors

\usepackage{float} % For forcing figure position
\usepackage{longtable} % For long tables

%% Define custom colors for the code
\definecolor{codegreen}{rgb}{0,0.6,0}
\definecolor{codegray}{rgb}{0.5,0.5,0.5}
\definecolor{codepurple}{rgb}{0.58,0,0.82}
\definecolor{backcolour}{rgb}{0.95,0.95,0.92}

% Define Python code style
\lstdefinestyle{mystyle}{
    backgroundcolor=\color{backcolour},   
    commentstyle=\color{codegreen},
    keywordstyle=\color{magenta},
    numberstyle=\tiny\color{codegray},
    stringstyle=\color{codepurple},
    basicstyle=\ttfamily\footnotesize,
    breakatwhitespace=false,         
    breaklines=true,                 
    captionpos=b,                    
    keepspaces=true,                 
    numbers=left,                    
    numbersep=5pt,                  
    showspaces=false,                
    showstringspaces=false,
    showtabs=false,                  
    tabsize=4
}

% Apply the style to Python listings
\lstset{style=mystyle}

\graphicspath{ {./figures/} }

%if you want smaller page margins, uncomment and adjust the following
%\textheight=24.3cm
%\topmargin=-1.8cm
%\textwidth=16.7cm
%\oddsidemargin=-0.3cm
%\evensidemargin=0.0cm


%Create your own environments
\newtheorem{definition}{Definition}
\newtheorem{example}{Example}
\newtheorem{theorem}{Theorem}

%and useful macrosfor faster writing (benefit: you can change the notations 
%later, e.g., two possible notations for your vector x)
\newcommand{\Mmatr}{\ensuremath{\mathbf{M}}}
\newcommand{\xvec}{\ensuremath{\overline{x}}}
\newcommand{\xvecII}{\ensuremath{\mathbf{x}}} %alternative def
\newcommand{\Xset}{\ensuremath{\mathbf{X}}}
\newcommand{\fr}{\ensuremath{\mathit{fr}}} %just neater typing

%If you want to remove the space before paragraphs uncomment the following.
%Remember then to leave an empty line between paragraphs! 
%\setlength{\parindent}{0pt}


\title{MDM-2024 Homework 5}
\author{Cuong Nguyen (101559968), Petteri Raita (909635),\\and Raihan Gafur (101555441)}

%Uncomment the following, if you don't want the date to be printed
%\date{}

\begin{document}

\maketitle
\tableofcontents
\newpage

\listoffigures
\newpage

% include tex files of sections here, e.g.
% \input{section1.tex}
% \input{section2.tex}
\section{preprocessing}


Describe briefly the preprocessing methods: tools (like \texttt{nltk}), in which order the steps were performed, stemmer, stopword list (including own additions), tf-idf version (equation), filtering (if any), and other possible steps or options that could affect the results.



The preprocessing of the language data started with a reading the CSV file and placing it in a Pandas dataframe with columns ID, Title, Abstract, and Text. The column text was a concatenation of the Title and the Abstract. The next steps will be applied on the combined column Text, since that takes into account both the Title and the Abstract, and both of those give information of how to cluster the documents.

The text is then tokenized using the NLTK library, which stands for the Natural Language Toolkit. After tokenization all text is turned into lowercase and numbers are removed. Next the punctuation is removed through comparing the text with the punctuation variable from the python base library string. For removing stopwords, we use the basic stopwords list from NLTK with option 'english' and also update the list with the ellipsis symbol. This step comprises all filtering that is done in preprocessing. The data is later filtered more with an expanded stopwords list based on the observations of the cluster digest terms.



The reason why the lowercasing and other preprocessing operations are done is that they, e.g., help matching words that only differ in their case. This makes it easier to find similarities between the texts.


After the the cleaning up of the tokens, the operation of stemming is applied. Stemming will shorten the words to their base or root form \autocite{aggarwal2015data}. Example of stemming for words "playing," "played," "player", which all get stemmed to their base form of "play". The stemmer used is the SnowballStemmer, which is an improved version of the Porter stemmer which itself is already one of the most widely used stemmers for English \autocite{porter1980algorithm}. After stemming all tokens get lemmatized which means that through a lookup table and morphological analysis, the words get turned into their base form. The difference with stemming is that lemmatization makes sure that the ouputted base form words are valid words.

\subsection{Order of preprocessing steps}

The first steps are logically placed for stemming or lemmatization to reduce the special symbols and stopwords. Their internal order is irrelevant since the deletion of numbers and punctuations are independent operations that do not affect each other. However, there is a fair question of lemmatization and stemming order. It can be argued that stemming first will reduce the available lemmatization and thus reducing the accuracy of the words. However, by stemming first we might enable better computational performance since lemmatization requires more operations per a single word. However, it is not certain if this is the optimal way to combine lemmatization and stemming. For future research, it would be interesting to see the difference of lemmatization only vs stemming + lemmatization for large corpora of data. Additional improvement ideas for the preprocessing could be the replacement of synonyms with a common synonym, thus reducing the distance between texts from authors that use words that have the same meaning but a different string representation.


\subsection{Vectorization}

The rerpresentation of the tokens is then turned into a matrix form, through the TfidVectorizer from the $sklearn.feature_extraction.text$ library. This function creates a matrix that has the documents in the rows and the columns include every unigram and a bigram which pass the required frequency we have set. The requirement for the n-grams to appear on columns is that they need to appear at most in 80\% of the documents and at least in 5 documents. This requirement will prune out words that don't have a high significance in distinguishing different documents from each other. The columns are then scaled to be unit vectors according to their L2 norm. This scaling ensures that the difference in absolute values of the columns don't impact the clustering.

The formula for TF-IDF used in the vectorizer:
\[
    \text{TF-IDF}(t, d, D) = \text{TF}(t, d) \times \text{IDF}(t, D)
\]

\[
    \text{TF}(t, d) = \frac{\text{Number of occurrences of term } t \text{ in document } d}{\text{Total number of terms in document } d}
\]

\[
    \text{IDF}(t, D) = \log\left(1 + \frac{N}{1 + n_t}\right)
\]


\section{K-Means Clustering and Topic Detection}

\subsection{Clustering Results}
The K-means clustering approach was employed to identify thematic groupings within the dataset. The optimal number of clusters $K$ was determined by minimizing the Davies-Bouldin Index (DBI), a metric that evaluates the quality of clustering by measuring intra-cluster similarity and inter-cluster separation. The best clustering result was achieved with $K = 6$, yielding a Davies-Bouldin Index of $6.18$. This indicates a high-quality clustering configuration with well-separated and compact clusters.

\subsection{Analysis of Unigrams and Bigrams}
To understand the thematic composition of the clusters, frequent unigrams and bigrams were extracted from each cluster. Text preprocessing steps included tokenization, removal of stopwords, stemming/lemmatization, and filtering out numeric and punctuation tokens. Table~\ref{tab:ngrams-clusters} summarizes the most frequent unigrams and bigrams for each cluster.

\begin{table}[H]
    \centering
    \begin{tabular}{|c|p{4cm}|p{6cm}|}
        \hline
        \multicolumn{1}{|c|}{\textbf{Cluster}} & \multicolumn{1}{c|}{\textbf{Topic}}               & \multicolumn{1}{c|}{\textbf{Top Unigrams and Bigrams}}                                                                                                                                         \\
        \hline
        1                                      & Programming Parallel Computers                    & compil, program, languag, code, graph, optim, memori, parallel, transform, implement, comput, processor, regist, theori, loop, design, system, type, applic, flow                              \\ \hline
        2                                      & Cryptography and Cloud Computing                  & secur, encrypt, scheme, key, cryptographi, attack, iot, protocol, data, propos, authent, imag, cloud, algorithm, cryptograph, chaotic, system, implement, devic, base                          \\ \hline
        3                                      & Robotic Systems for Human Centric Tasks           & robot, control, soft, system, task, environ, learn, simul, human, model, actuat, perform, materi, optim, develop, soft robot, manipul, sensor, evolv, approach                                 \\ \hline
        4                                      & Query Optimization in Relational Database Systems & databas, data, queri, relat, relat databas, system, sql, ontolog, inform, model, schema, approach, manag, process, store, graph, inform system, user, web, semant                              \\ \hline
        5                                      & Computer Vision in Deep Network Systems           & imag, detect, vision, method, comput vision, model, learn, object, comput, deep, network, propos, system, video, track, featur, dataset, visual, accuraci, perform                             \\ \hline
        6                                      & Cryptography in Quantum Computing Systems         & quantum, quantum comput, gate, secur, comput, compil, protocol, cryptographi, qubit, circuit, post quantum, key, attack, algorithm, post, oper, state, communic, quantum cryptographi, program \\ \hline
    \end{tabular}
    \caption{Frequent unigrams and bigrams for each cluster.}
    \label{tab:ngrams-clusters}
\end{table}

The frequent terms and phrases highlight the thematic focus of each cluster. For instance, Cluster 1 is characterized by discussions on ``Programming Parallel Computers'', while Cluster 2 focuses on ``Cryptography and Cloud Computing''. Similar thematic insights were derived for other clusters.

\section{Additional Experiments}
\subsection{Latent Semantic Analysis (LSA)}
Following the tf-idf transformation, the resulting matrix comprises 3,828
features across 1,143 documents. Given the high dimensionality of the data,
applying Latent Semantic Analysis (LSA) is a logical step for dimensionality
reduction. For this purpose, we employed the \texttt{TruncatedSVD} class from
the \texttt{sklearn.decomposition} module. As a general guideline, the number
of components for LSA typically ranges between 100 and 400, depending on the
dataset's size.

To identify the optimal number of components, a grid search was conducted
using the same experimental settings as the baseline model. The analysis
determined that 100 components yielded the best results, as depicted in
\autoref{fig:optimal_n_component}. The corresponding Davies-Bouldin score of
3.0 is approximately half that of the baseline model, demonstrating that LSA
effectively reduces data dimensionality and enhances clustering performance.
Furthermore, the associated number of clusters is seven.

Building on the LSA-reduced feature matrix, further improvements in clustering
performance can be achieved by refining the initialization of k-means
centroids. This approach is discussed in the subsequent section.

\begin{figure}[H]
    \centering
    \includegraphics[height=\textheight,width=\textwidth,keepaspectratio]%
    {optimal_n_component.png}
    \caption{Davies-Bouldin scores for varying numbers of components in LSA.}
    \label{fig:optimal_n_component}
\end{figure}

\subsection{Scatter/Gather --- Buckshot Initialization}
In the baseline model, the \emph{k-means++} initialization method was employed
to initialize the centroids. While widely used, this approach does not always
guarantee optimal performance, as the initial seed selection may lead to
convergence at a local minimum. To address this limitation, Buckshot
initialization combined with Agglomerative Clustering was utilized.

The Buckshot initialization method involves randomly sampling a subset of data
with a size of $\sqrt{K \cdot n}$, where $K$ represents the number of clusters
and $n$ is the total number of data points. This subset is then clustered
using Agglomerative Clustering, and the resulting centroids are used as the
initial seeds for the k-means algorithm. With this initialization, the
Davies-Bouldin score improved to 2.67, outperforming the LSA-reduced model. The
associated number of clusters in this configuration is 8, as illustrated in
\autoref{fig:optimal_k_clusters}.

\begin{figure}[H]
    \centering
    \includegraphics[height=\textheight,width=\textwidth,keepaspectratio]%
    {optimal_k_clusters.png}
    \caption{Davies-Bouldin scores for varying numbers of clusters in the
        Scatter/Gather Buckshot-initialized model.}
    \label{fig:optimal_k_clusters}
\end{figure}

\subsection{Concluded Topics}
The clustering analysis facilitated the derivation of distinct topics by
examining the top 20 most common terms within each cluster. The identified
topics are summarized in \autoref{tab:concluded_topics}.


\begin{longtable}{|p{3cm}|p{7cm}|}
    \hline
    \textbf{Topic}                                                & \textbf{Top 20 most common terms in each cluster digest             }                                                                                                        \\
    \hline
    Robotics and Computer Vision with Machine Learning Approaches & robot, method, imag, system, model, propos, detect, perform, learn, comput, result, base, data, algorithm, vision, network, approach, object, studi, develop                 \\
    \hline
    Quantum Computing and Cryptography                            & quantum, comput, secur, compil, algorithm, cryptographi, oper, propos, key, implement, protocol, attack, gate, program, state, circuit, scheme, communic, post-quantum, time \\
    \hline
    Programming Languages and Code Optimization                   & compil, program, code, languag, comput, optim, system, graph, implement, memori, paper, parallel, design, applic, perform, transform, algorithm, approach, model, base       \\
    \hline
    Database Query Systems                                        & queri, databas, data, sql, relat, languag, system, approach, model, process, execut, user, translat, propos, natur, base, specul, paper, result, algorithm                   \\
    \hline
    Data Security and Cryptographic Applications in IoT           & secur, encrypt, propos, data, scheme, key, cryptographi, system, base, algorithm, imag, attack, protocol, comput, implement, iot, authent, provid, applic, paper             \\
    \hline
    Mathematical Optimization and Algorithm                       & optim, ti, new, general, triangular, inequ, problem, techniqu, ti-bas, deploy, strength, reduct, algorithm, compil, name, form, x, averag, theori, compiler-bas              \\
    \hline
    Database Management Systems and Data Modeling                 & databas, data, relat, system, inform, model, approach, propos, process, manag, ontolog, paper, develop, design, applic, schema, method, base, result, perform                \\
    \hline
    Type Systems and Static Analysis for Programming Languages    & type, qualifi, checker, java, framework, programm, system, compil, write, plug-in, error, custom, type-check, languag, semant, code, detect, backward-compat, way, program   \\
    \hline
    \caption{Concluded topics and their top 20 most common terms in each cluster digest.}
    \label{tab:concluded_topics}
\end{longtable}



%Bibliography style. The alpha style generates references with 
%first letters and year. If you prefer numbers, use style plain.
% \bibliographystyle{alpha}
% \bibliographystyle{plain}
% \bibliography{ref}

\printbibliography{}

% Demo for adding appendices
\begin{appendices}
    All code is publicly available on GitHub at \url{https://github.com/ancuongnguyen07/CS-E4650/tree/master/hw5}
\end{appendices}


\end{document}
