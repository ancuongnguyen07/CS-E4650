\section{K-Means Clustering and Topic Detection}

\subsection{Clustering Results}
The K-means clustering approach was employed to identify thematic groupings within the dataset. The optimal number of clusters $K$ was determined by minimizing the Davies-Bouldin Index (DBI), a metric that evaluates the quality of clustering by measuring intra-cluster similarity and inter-cluster separation. The best clustering result was achieved with $K = 6$, yielding a Davies-Bouldin Index of $6.18$. This indicates a high-quality clustering configuration with well-separated and compact clusters.

\subsection{Analysis of Unigrams and Bigrams}
To understand the thematic composition of the clusters, frequent unigrams and bigrams were extracted from each cluster. Text preprocessing steps included tokenization, removal of stopwords, stemming/lemmatization, and filtering out numeric and punctuation tokens. Table~\ref{tab:ngrams-clusters} summarizes the most frequent unigrams and bigrams for each cluster.

\begin{table}[h!]
\centering
\begin{tabular}{|c|p{10cm}|}
\hline
\textbf{Cluster} & \textbf{Top Unigrams and Bigrams} \\ \hline
Cluster 1        & compil, program, languag, code, graph, optim, memori, parallel, transform, implement, comput, processor, regist, theori, loop, design, system, type, applic, flow \\ \hline
Cluster 2        & secur, encrypt, scheme, key, cryptographi, attack, iot, protocol, data, propos, authent, imag, cloud, algorithm, cryptograph, chaotic, system, implement, devic, base \\ \hline
Cluster 3        & robot, control, soft, system, task, environ, learn, simul, human, model, actuat, perform, materi, optim, develop, soft robot, manipul, sensor, evolv, approach \\ \hline
Cluster 4        & databas, data, queri, relat, relat databas, system, sql, ontolog, inform, model, schema, approach, manag, process, store, graph, inform system, user, web, semant \\ \hline
Cluster 5        & imag, detect, vision, method, comput vision, model, learn, object, comput, deep, network, propos, system, video, track, featur, dataset, visual, accuraci, perform \\ \hline
Cluster 6        & quantum, quantum comput, gate, secur, comput, compil, protocol, cryptographi, qubit, circuit, post quantum, key, attack, algorithm, post, oper, state, communic, quantum cryptographi, program \\ \hline
\end{tabular}
\caption{Frequent unigrams and bigrams for each cluster.}
\label{tab:ngrams-clusters}
\end{table}

The frequent terms and phrases highlight the thematic focus of each cluster. For instance, Cluster 1 is characterized by discussions on ``Programming Parallel Computers'', while Cluster 2 focuses on ``Cryptography and Cloud Computing''. Similar thematic insights were derived for other clusters.

\subsection{Conclusion on Topics}
The clustering results reveal coherent thematic groupings that align with distinct research areas. Frequent terms within the clusters provide valuable insights into the dominant topics. For example:
\begin{itemize}
    \item \textbf{Cluster 1:} Programming Parallel Computers
    \item \textbf{Cluster 2:} Cryptography and Cloud Computing
    \item \textbf{Cluster 3:} Soft Robotic Systems for Human Centric Tasks
    \item \textbf{Cluster 4:} Query Optimization in Relational Database Systems
    \item \textbf{Cluster 5:} Computer Vision in Deep Network Systems
    \item \textbf{Cluster 6:} Cryptography in Quantum Computing Systems
\end{itemize}

Overall, the K-means approach successfully identified meaningful topics within the dataset, providing a clear categorization of the underlying research themes. These results demonstrate the utility of K-means clustering in topic detection and thematic analysis.